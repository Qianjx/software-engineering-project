\chapter{数据结构设计}
\section{逻辑结构设计}
\subsection{用户管理系统数据结构设计}
讲述本系统内需要什么数据结构。这指的是程序运行过程中维护的数据结构。只是举个例子,此处应和3.3一致。
\subsection{客户端数据结构}

\subsection{用户端数据结构}

\section{物理结构设计}
各数据结构无特殊物理结构要求。(如果有,比如说hadoop等,应当具体说明)

\section{数据结构与程序模块的关系}
[此处指的是不同的数据结构分配到哪些模块去实现。可按不同的端拆分此表]
\begin{table}[htbp]
\centering
\caption{数据结构与程序代码的关系表} \label{tab:datastructure-module}
\begin{tabular}{|c|c|c|c|}
    \hline
    · & 模块1 & 模块2 & 模块3 \\
    \hline
    结构1 & · & Y & · \\
    \hline
    结构2 & · & Y & · \\
    \hline
    结构3 & · & Y & · \\
    \hline
    结构4 & Y & · & · \\
    \hline
    结构5 & · & · & Y \\
    \hline
\end{tabular}
\note{各项数据结构的实现与各个程序模块的分配关系}
\end{table}