\chapter{任务概述}
本系统的目标是实现一个即时通信系统,包括客户端、服务器端两个部分。

客户端面向Android或windows用户,为用户提供即时的文字、语言聊天,群聊等功能

\section{目标}
实现实时通信系统,实现需求规格说明书中所描述的功能需求、性能需求,以及实现实时聊天功能、实时语音视频功能和群聊功能,并且保证系统的健壮性和数据安全。

\section{开发与运行环境}

\subsection{开发环境的配置}
\begin{table}[htbp]
\centering
\caption{开发环境的配置} \label{tab:development-environment}
\begin{tabular}{|c|c|c|}
    \hline
    类别 & 标准配置 & 最低配置 \\
    \hline
    计算机硬件 & \tabincell{c}{基于x86结构的CPU\\ 主频>=2.4GHz\\ 内存>=8G\\ 硬盘>=200G} & \tabincell{c}{基于x86结构的CPU\\ 主频>=1.6GHz\\ 内存>=512M\\ 硬盘>=2G} \\
    \hline
    计算机软件 & \tabincell{c}{Linux (kernel version>=4.10)\\ GNU gcc (version>=6.3.1)} & \tabincell{c}{Linux (kernel version>=3.10)\\ GNU gcc (version>=5.4)} \\
    \hline
    网络通信 & \tabincell{c}{至少要有一块可用网卡\\ 能运行IP协议栈即可} & \tabincell{c}{至少要有一块可用网卡\\ 能运行IP协议栈即可} \\
    \hline
    其他 & 采用MySQL数据库 & 采用MySQL数据库 \\
    \hline
\end{tabular}
% \note{这里是表的注释}
\end{table}

\subsection{测试环境的配置}
\begin{table}[htbp]
\centering
\caption{测试环境的配置} \label{tab:test-environment}
\begin{tabular}{|c|c|c|}
    \hline
    类别 & 标准配置 & 最低配置 \\
    \hline
    计算机硬件 & \tabincell{c}{基于x86结构的CPU\\ 主频>=2.4GHz\\ 内存>=8G\\ 硬盘>=200G} & \tabincell{c}{基于x86结构的CPU\\ 主频>=1.6GHz\\ 内存>=512M\\ 硬盘>=2G} \\
    \hline
    计算机软件 & \tabincell{c}{Linux (kernel version>=4.10)\\ GNU gcc (version>=6.3.1)} & \tabincell{c}{Linux (kernel version>=3.10)\\ GNU gcc (version>=5.4)} \\
    \hline
    网络通信 & \tabincell{c}{至少要有一块可用网卡\\ 能运行IP协议栈即可} & \tabincell{c}{至少要有一块可用网卡\\ 能运行IP协议栈即可} \\
    \hline
    其他 & 采用MySQL数据库 & 采用MySQL数据库 \\
    \hline

\end{tabular}
% \note{这里是表的注释}
\end{table}

\subsection{运行环境的配置}
\begin{table}[htbp]
\centering
\caption{运行环境的配置} \label{tab:operation-environment}
\begin{tabular}{|c|c|c|}
    \hline
    类别 & 标准配置 & 最低配置 \\
    \hline
    计算机硬件 & \tabincell{c}{基于x86结构的CPU\\ 主频>=2.4GHz\\ 内存>=8G\\ 硬盘>=200G} & \tabincell{c}{基于x86结构的CPU\\ 主频>=1.6GHz\\ 内存>=512M\\ 硬盘>=2G} \\
    \hline
    计算机软件 & \tabincell{c}{Linux (kernel version>=4.10)\\ GNU gcc (version>=6.3.1)} & \tabincell{c}{Linux (kernel version>=3.10)\\ GNU gcc (version>=5.4)} \\
    \hline
    网络通信 & \tabincell{c}{至少要有一块可用网卡\\ 能运行IP协议栈即可} & \tabincell{c}{至少要有一块可用网卡\\ 能运行IP协议栈即可} \\
    \hline
    其他 & 采用MySQL数据库 & 采用MySQL数据库 \\
    \hline

\end{tabular}
% \note{这里是表的注释}
\end{table}

\section{需求概述}
功能需求包括:
通讯模式
– 用户一对一通讯
– 群聊
– 随机匹配聊天
• 通讯方式
– 文字即时通讯
– 图片、视频、音频、链接、文件等多种媒体的发送
– 语音、视频及时通讯
– 加密文字及时通讯
• 用户控制
– 注册/注销账号
– 添加/删除联系人
– 加入/退出/创建群聊

\section{条件与限制}
本节至少要与需求说明文档中相关章节相一致。

4.2 硬件约束
1. Android 平台:软件本身大小 <100MB;本地聊天记录文件不大于 1G,并
且有着自动清除过时的聊天记录功能;文本通信延迟 <100ms;内存占用
<100MB;流量消耗不超过通信本身数据量的 1.1 倍;电量消耗不超过机体
待机时的两倍
2. Windoes 平台:约束较少,主要考虑的磁盘和内存约束与 Android 一致

4.3 技术限制
1. 考虑到服务器网络带宽问题,视频聊天的清晰度最高维持在 360P;
2. 由于技术上的限制,除去文本类的通信记录都不会被保存;
3. 用户无法在不同的客户端上共享聊天记录;
4. 用户无法在周日凌晨 2:00 至 4:00 使用通讯软件的聊天功能因为这个时
候应该是维护服务器的时候;
5. 由于服务器磁盘大小的限制,用户传输文件的大小有着 <8MB 的限制
